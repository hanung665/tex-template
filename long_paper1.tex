\documentclass[a4paper, 12pt, oneside]{book}
\usepackage[bahasa]{babel}
\usepackage[a4paper, left=2.5cm, right=2cm, top=2.5cm, bottom=2.5cm]{geometry}
\usepackage{lipsum}
\usepackage{graphicx}

%%%%%
\usepackage{setspace}
    \doublespace
    %singlespace
    %onehalfspace
\usepackage{titlesec}
    \titleformat*{\section}{\bf}
    \titleformat*{\subsection}{\bf}
\usepackage{indentfirst}
\parindent=1.45cm
\usepackage{float} % biar bisa force here buat gambar dan tabel
%\usepackage[colorlinks=true,linkcolor=red]{hyperref}
\usepackage{fancyhdr}
    \pagestyle{fancy}
    \chead{}
    \rhead{}
    \lhead{}
    \rfoot{}
    \lfoot{}
    \cfoot{\thepage}
    \renewcommand{\headrulewidth}{0.0pt} %hapus garis di header
    \renewcommand{\footrulewidth}{0.4pt} % garisnya dipindah ke sini

%%%%%
%\usepackage{hyphenation} %setelah kelar nulis baru cek pemenggalan kata, yang ga bener baru dikoreksi di sini
    %\hyphenation{mung-kin, sa-ngar, mi-num}
    
\makeatletter
    \renewcommand{\@makechapterhead}[1]{
    \vspace*{0\p@}%
    {\parindent \z@ \raggedright \normalfont
    \ifnum \c@secnumdepth >\m@ne
      \if@mainmatter
      \center \MakeUppercase{\normalsize\bfseries \@chapapp\space \thechapter}
      %\chapapp\space memunculkan kata "Bab"
        \par\nobreak
        \vskip 10\p@ % Mengatur jarak antara "Bab 1" dan "Pendahuluan"
      \fi
    \fi
    \interlinepenalty\@M
    \normalsize \uppercase{\bfseries #1\par\nobreak} %Bagian ini memunculkan judul bab
    \vskip 30\p@ % Mengatur jarak antara "Bab 1" dengan "Teks"
    }}
\makeatother

%%%%%
%\usepackage{color}
%Teks ini merah dan yang ini biru Kode:
%    \textcolor{red}{Teks ini merah}\ \textcolor{blue}{dan yang ini biru}
%\newcommand{\merah}[1]{\textcolor{red}{#1}}

%%%%%
% Kasih border hitam sekeliling gambar
%\setlength\fboxsep{0pt}
%    \setlength\fboxrule{2pt} % bagian ini untuk merubah tebal garis
%    \fbox{\includegraphics{bab}}

%%%
%baris 1
%\vspace{\baselineskip}
%baris 2

%%%%%
% Buat nulis code snippets
\usepackage{listings}
\lstset{
    language=Python,
    basicstyle=\ttfamily,
    numbers=left,
    numberstyle=\tiny\color{gray},
    xleftmargin=15pt,
    showstringspaces=false
}

%%%%%
% Setting kedalaman section dan ToC section hingga 5 level
%\setcounter{secnumdepth}{5}
%\setcounter{tocdepth}{5}

%%%%%
% ubah ukuran font verbatim typewriter
%\makeatletter
%\g@addto@macro\@verbatim\footnotesize
%\makeatother

%%%%%
% Lebar kolom fixed x cm

%\usepackage{array}
%\newcolumntype{L}[1]{>{\raggedright\let\newline\\\arraybackslash\hspace{0pt}}m{#1}}
%\newcolumntype{C}[1]{>{\centering\let\newline\\\arraybackslash\hspace{0pt}}m{#1}}
%\newcolumntype{R}[1]{>{\raggedleft\let\newline\\\arraybackslash\hspace{0pt}}m{#1}}
%
%\begin{document}
%
%\begin{tabular}{| c | L{3cm} | C{3cm} | R{3cm} |}
%foo &
%A cell with text that wraps around, is raggedright and allows \newline
%    manual line breaks &
%A cell with text that wraps around, is centered and allows \newline
%    manual line breaks &
%A cell with text that wraps around, is raggedleft and allows \newline
%    manual line breaks \\
%\end{tabular}

%\end{document}
%%%%%

\begin{document}

%nopagenumbers

\pagenumbering{roman}
%\maketitle
\begin{titlepage}
\begin{center}
%\fontsize{20pt}{20pt}\selectfont\bfseries
%\linespread{1}\Large\bfseries
\LARGE \textbf {RANCANG BANGUN SISTEM INFORMASI PENJEMPUTAN SISWA SEKOLAH}
\end{center}

\begin{center}
{\Large (Studi Kasus SDN 1 Jombang)}
\vspace{1cm}\\
\end{center}

\begin{center}
{\large TUGAS AKHIR}
\vspace{1cm}\\
\large{Diajukan Untuk Memenuhi Salah Satu Syarat Dalam Menempuh Tugas Akhir Program Studi Sistem Informasi}
\vfill
\includegraphics[height=7cm]{logo.png}
\vfill
{\large Hanung Adi Wijaya\\
1813070043}
%\vspace{0.5in}\\
%Under the direction of:\\
%Dr. Super Famous Person\\
%Massachusetts Institute of Technology
\vspace{1cm}\\
\large{
PROGRAM STUDI SISTEM INFORMASI\\
JURUSAN TEKNIK INFORMATIKA\\
PERBANAS INSTITUTE\\
2019}
%\today
\end{center}

\end{titlepage}
\frontmatter
\subsection*{Ucapan Terima Kasih}
\addcontentsline{toc}{subsection}{Ucapan Terima Kasih}
Isi halaman preface / foreword / pengantar / terima kasih

\newpage
\section*{Lembar Pengesahan}
\addcontentsline{toc}{subsection}{Lembar Pengesahan}
Isi halaman lembar pengesahan

\tableofcontents
\listoffigures
\listoftables
\newpage
%\chapter{ABSTRACT}
%\begin{abstract}
\begin{center}
%\linespread{2}
\textbf{ABSTRACT}\\
STUDENT PICK UP INFORMATION SYSTEM\\
(Case Study SDN 1 Jombang)\\
Hanung Adi Wijaya\\
\vspace{1cm}
\end{center}
{\sf \lipsum[1]}\\
\lipsum[2]\\
\lipsum[3]\\
%\end{abstract}

%\newpage
\pagenumbering{arabic}
\mainmatter
\chapter{Pendahuluan}
\section{Latar Belakang}

%\begin{abstract}
%\input{abstract}
%\vspace{1in}
%\input\{Summary}
%\end{abstract}
\subsection{Sosiokultural}
%\subsubsection{Masyarakat}
%\paragraph{\emph{\lipsum[1]}}
%\subparagraph{\textit{\lipsum[1]}}
\textsl{\lipsum[1]}

rm:

\textrm{\lipsum[4]}

No formatting:

\lipsum[4]

Tiny:

{\tiny \lipsum[4]}

Scriptsize:

{\scriptsize \lipsum[4]}

Footnote size:

{\footnotesize \lipsum[4]}

Small:

{\small \lipsum[4]}

Normal size:

{\normalsize \lipsum[4]}

large:

{\large \lipsum[4]}

Large:

{\Large \lipsum[4]}

LARGE:

{\LARGE \lipsum[4]}

huge:

{\huge \lipsum[4]}

Huge:

{\Huge\lipsum[4]}

\textsf{SF: \lipsum[5]}

\textit{IT: \lipsum[6]}

\textsc{SC: \lipsum[7]}

\vspace{1cm}
\begin{flushright}
TT:

%\begin{quote}
\texttt{\lipsum[8]}
%\end{quote}

\begin{enumerate}
\item Ini item satu.
\item Ini item dua.

Ini bagian dari item dua.
\item Ini item ketiga.
\item dan ini item terakhir
\end{enumerate}
\end{flushright}
Carrots are good for your eyes \cite{les85}, since they contain Vitamin A\@. Have you ever seen a rabbit wearing glasses?  The numbers 1, 2, 3, etc.\ are called \lq natural numbers\rq. According to Kronecker, they were made by \lq\lq God\rq\rq; all else being the work of Man. And this is sample of footnote\footnote{\label{myfootnote}This footnote is sponsored by DMR Project}.

And I'm referring to footnote \ref{myfootnote}, said that this is should be awesome. \lipsum[4]

\lipsum[5]
\lipsum[6]
%\newpage
\chapter{Landasan Teori}
\lipsum[7-15]
\chapter{Metode Penelitian}
\section{Pengumpulan Data}
\lipsum[16-21]
\chapter{Pembahasan}
\lipsum[22-30]
\chapter{Kesimpulan}
\lipsum[31-35]
\newpage
\begin{glossary}
Kholid Fuadi, dkk., {\it Panduan Menulis Tesis dengan \LaTeX\
edisi I} (Yogyakarta: Banteng Press, 2011), hlm. 176. 
\end{glossary}
%\begin{thereference}
%\end{thereference}
\begin{thebibliography}{10}
\bibitem{les85}Leslie Lamport, 1985. \emph{\LaTeX---A Document Preparation System---User’s Guide and Reference Manual}, Addision-Wesley, Reading.
\bibitem{don89}Donald E. Knuth, 1989. \emph{Typesetting Concrete Mathematics}, TUGBoat, 10(1):31-36.
\bibitem{rondon89}Ronald L. Graham, Donald E. Knuth, and Ore Patashnik, 1989. \emph{Concrete Mathematics: A Foundation for Computer Science}, Addison-Wesley, Reading.
\bibitem{kholid}Kholid Fuadi, dkk., {\it Panduan Menulis Tesis dengan \LaTeX\
edisi I} (Yogyakarta: Banteng Press, 2011), hlm. 176. 
\end{thebibliography}
\end{document}